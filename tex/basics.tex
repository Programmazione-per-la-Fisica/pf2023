% !TeX root = ./pf2023.tex

%\includeonlyframes{current}

\section*{Basics}

\begin{frame}{C++}

  \begin{itemize}[<+->]
  \item There are many programming languages, with very different characteristics
  \item Why \Cpp{}?
    \begin{itemize}[<.->]
    \item general purpose
    \item support for multiple styles of programming (\textit{paradigms})
    \item much used in scientific fields, but also in games, finance,
      telecommunications, embedded, \ldots
    \item available on most architectures and operating systems
    \item efficient
    \item ISO standard
    \end{itemize}
  \end{itemize}
\end{frame}

\begin{frame}[fragile]{A minimal C++ program}

  \begin{itemize}
  \item This program does nothing, successfully (details will be explained)
    \begin{codeblock}
int main() \{\}\end{codeblock}

  \item<2-> Let's write these 13 characters (including spaces) in a \textbf{text
      document}, which we call \code{minimal.cpp}, using Visual Studio Code
    \begin{itemize}
    \item In \Cpp{} terms, this text document is called \alert{source file}
    \item Also, let's start using the \textit{shell}, writing commands in a
      \textit{Terminal} instead of clicking icons with the mouse
    \end{itemize}

    \begin{shellblock}<3->{
\$\tikzref{prompt}
\uncover<4->{\$ code minimal.cpp \inserthitenter\tikzref{command}}
\uncover<6->{\$ cat minimal.cpp \inserthitenter\tikzref{cat}
int main() \{\}
\$}}\end{shellblock}

\uncover<3>{\tikz[remember picture,overlay] \draw[<-,white] ([xshift=1ex] prompt) -- +(.5,0) node[white,right] {\scriptsize this is the shell prompt};}
\uncover<4>{\tikz[remember picture,overlay] \draw[<-,white] ([xshift=1ex] command) -- +(.5,0) node[white,right] {\scriptsize write this command (hit ``Enter'' at the end)};}
\uncover<6>{\tikz[remember picture,overlay] \draw[<-,white] ([xshift=1ex] cat) -- +(.5,0) node[white,right] {\scriptsize the cat command shows the contents of a text file};}

  \item<7> Before \alert{executing} the program we need to translate it into
    the language of the computer
  
  \end{itemize}

\end{frame}

\begin{frame}[fragile]{A minimal C++ program \insertcontinuationtext}

  \begin{itemize}

  \item In C++ the translation into a binary format is the job of the
    \alert{compiler}, which produces an \alert{executable} file

\begin{center}
    \begin{tikzpicture}[every text node part/.style={align=center}]
      \node[tape,tape bend height=.3cm,draw,thick,inner sep=2pt,minimum height=2cm,minimum width=3cm] (source) {Source file \\ {\scriptsize(\code{minimal.cpp})}};
      \node[tape,tape bend height=.3cm,black,draw,thick, inner sep=2pt,minimum height=2cm,minimum width=3cm,right=3cm of source] (exe) {Executable binary file \\ {\scriptsize(\code{a.out})}};
      \draw[->] (source) -- (exe) node[above,align=center,midway] {Compilation \\ {\scriptsize(\code{g++})}};
    \end{tikzpicture}
  \end{center}

  \begin{shellblock}<2->{
\$ g++ minimal.cpp
\$ \uncover<3->{ls\tikzref{ls}
a.out  minimal.cpp
\$ }\uncover<4->{./a.out\tikzref{program}
\$ }\uncover<5->{g++ minimal.cpp -o minimal
\$ ls
a.out  minimal  minimal.cpp
\$ }\uncover<6->{./minimal
\$ }}\end{shellblock}

  \uncover<3>{\tikz[remember picture,overlay] \draw[<-,white] ([xshift=1ex] ls) -- +(1,0) node[white,right] {\scriptsize the ls command lists the contents of a directory};}
  \uncover<4>{\tikz[remember picture,overlay] \draw[<-,white] ([xshift=1ex] program) -- +(1,0) node[white,right] {\scriptsize this command executes the program};}

  \item<7-> The \code{-o} option passed to the \code{g++} command allows to give
    another name to the final executable
  \end{itemize}

\end{frame}

\begin{frame}[fragile]{Spaces}

  \begin{itemize}
  \item Spaces are (almost) irrelevant
  \end{itemize}

  \begin{columns}[T]
  \begin{column}{.3\textwidth}<2->
    \begin{codeblock}
int main()\{\}

\end{codeblock}
  \end{column}

  \begin{column}{.3\textwidth}<3->
    \begin{codeblock}
int
main(  ) \{
   \}\end{codeblock}
  \end{column}

  \begin{column}{.3\textwidth}<4->
    \begin{codeblock}
int\tikzref{n1}main()\{ \}

    \end{codeblock}
  \end{column}

  \uncover<5->{
\begin{tikzpicture}[remember picture,overlay]
  \node[circle,radius=7pt,draw,red,line width=1pt,opacity=.7] at (n1) (n2){};
  \node[above=of n2,draw=red,line width=1pt,opacity=.7,rounded corners] {space needed here} edge[->,line width=1pt,red,opacity=.7] (n2);
\end{tikzpicture}
  }

\end{columns}

\begin{itemize}[<6->]
\item Tools exist to consistently format source code to improve readability
  \begin{itemize}
  \item They are customizable
  \item You are \textbf{required} to use them
  \end{itemize}
\end{itemize}

\begin{columns}[T]
  \begin{column}{.4\textwidth}
    \begin{shellblock}<7->{
\$ clang-format minimal.cpp
int main()
\{
\}
\$}\end{shellblock}
  \end{column}

  \begin{column}{.6\textwidth}
  \begin{codeblock}<8->{
# .clang-format file
\ldots
AllowShortFunctionsOnASingleLine: false
BraceWrapping:
  AfterFunction:   true
  \ldots}\end{codeblock}
  \end{column}
\end{columns}

\begin{itemize}[<9->]
\item \href{https://isocpp.github.io/CppCoreGuidelines/CppCoreGuidelines#nl4-maintain-a-consistent-indentation-style}{NL.4}
\end{itemize}

\end{frame}

\begin{frame}[fragile]{Syntax check}

  \begin{itemize}
  \item The compiler can translate a program into a binary executable only if the
    code is syntactically correct

  \begin{codeblock}<2->{
intmain() \{\}}\end{codeblock}

  \begin{shellblock}<2->{
\$ g++ minimal.cpp
minimal.cpp:1:9: error: ISO C++ forbids declaration of \upquote{intmain} with \ldots
    1 | intmain() \{\}
      |         ^
\$}\end{shellblock}

  \item<3-> Error messages are usually precise about the cause of the error, but
    not always
  \item<3-> Learn to interpret error messages
  \end{itemize}

\end{frame}

\begin{frame}[fragile]{Comments}

  \begin{itemize}

  \item Code can contain comments

    \begin{codeblock}<2->{
int main() \alert{//} comment until the end of the line
\{
\}}\end{codeblock}

    \begin{codeblock}<3->{
int main()
\{
  \alert{/*} possibly multi-line comment
     that goes until the final marker
     and cannot nest \alert{*/}
\}}\end{codeblock}

  \item<4-> Comments are ignored by the compiler and are equivalent to spaces
  \item<4-> Comments are for humans
    \begin{itemize}
    \item Prefer expressive code
    \item
      \href{https://isocpp.github.io/CppCoreGuidelines/CppCoreGuidelines#nl1-dont-say-in-comments-what-can-be-clearly-stated-in-code}{NL.1},
      \href{https://isocpp.github.io/CppCoreGuidelines/CppCoreGuidelines#nl2-state-intent-in-comments}{NL.2},
      \href{https://isocpp.github.io/CppCoreGuidelines/CppCoreGuidelines#nl3-keep-comments-crisp}{NL.3}
    \end{itemize}
  \end{itemize}

\end{frame}

\begin{frame}[fragile]{Basic notions of Input and Output}

  \begin{itemize}
  \item The input and output system allows a program to interact with the
    external world
  \item One mechanism offered by C++ to do input and output is the \textit{I/O
      streams} interface, which allows the creation and the manipulation of
    \textit{stream} entities
    \begin{itemize}
    \item Values are extracted from an input stream
    \item Values are inserted into an output stream
    \end{itemize}
  \item Input and output streams connected to the terminal are automatically
    available to a program
    \begin{itemize}
    \item \code{std::cin}, \code{std::cout} (and \code{std::cerr} for errors)
    \end{itemize}
  \item To extract and insert values, use the stream \textit{operators}
    \code{>>} and \code{<<}, respectively
  \end{itemize}

\end{frame}

\begin{frame}[fragile]{Hello}

  A less minimal C++ program (details will be explained)

  \begin{codeblock}
\alert<2|trans:0>{#include <iostream>}\uncover<2->{ // import I/O utilities}
\alert<2|trans:0>{#include <string>}\uncover<2->{ // import string utilities}

int \alert<3|trans:0>{main}()\uncover<3->{ // start the program from here}
\{
\alert<4|trans:0>{  std::cout << "What\textquotesingle{}s your name? ";}\uncover<4->{ // print a message on the terminal}
\alert<5|trans:0>{  std::string name;}\uncover<5->{ // some space is needed in memory for a string}
\alert<6|trans:0>{  std::cin >> name;}\uncover<6->{ // read a string from the terminal into that space}
\alert<7|trans:0>{  std::cout << "Hello, " << name << \bslashn{};}\uncover<7->{ // print a multi-part message}
\}\end{codeblock}

\begin{shellblock}<8->{
\uncover<8->{\$ g++ hello.cpp -o hello \inserthitenter
\$ }\uncover<9->{./hello \inserthitenter
What\textquotesingle{}s your name? }\uncover<10->{Francesco \inserthitenter
Hello, Francesco
\$ }}\end{shellblock}

\end{frame}

\begin{frame}{Objects}

  \begin{itemize}[<+->]
  \item The constructs in a C++ program create, destroy, refer to, access,
    and manipulate \alert<.>{objects}
  \item An object is a region of storage (i.e. memory)
    \begin{itemize}[<+->]
    \item it has a \alert<.>{type}
    \item it has a \alert<.>{lifetime}
    \item it can have a \alert<.>{name}
    \end{itemize}
  \end{itemize}

\end{frame}

\begin{frame}{Types}

  \begin{itemize}[<+->]
  \item A type gives meaning to a piece of storage
    \begin{itemize}
    \item What's the meaning of a piece of storage that contains the sequence of
      bits $01100011011010010110000101101111$?
    \item Read as a sequence of alphabetic characters, it's the letters
      \textit{c}, \textit{i}, \textit{a}, \textit{o}
    \item Read as an integer number, it's $1667850607$
    \item Read as a floating-point number, it's about $4.30511 \times 10^{21}$
    \end{itemize}
  \item A type identifies a set of values and the operations that can be applied
    to those values
    \begin{itemize}
    \item C++ is a \alert{strongly typed} language (mostly)
    \end{itemize}
  \item A type is also associated with a machine representation for the values
    belonging to the type
  \end{itemize}

\end{frame}

\begin{frame}{Types \insertcontinuationtext}

  \begin{itemize}[<+->]
  \item The compiler checks that program instructions are compatible with the type system
    \begin{itemize}
    \item C++ is a \alert{statically typed} language (mostly)
    \end{itemize}
  \item C++ defines a few fundamental types and provides mechanisms to build
    compound types on top of them
  \end{itemize}

\end{frame}

\begin{frame}{Fundamental types}
  \begin{itemize}
    \item arithmetic types
      \begin{itemize}
        \item integral types
          \begin{itemize}
          \item signed integer types: \code{short int}, \alert<3->{\code{int}},
            \code{long int}, \code{long long int}
          \item unsigned integer types: \code{unsigned short int},
            \code{unsigned int}, \code{unsigned long int}, \code{unsigned long
              long int}
          \item character types: \alert<3->{\code{char}}, \code{signed char},
            \code{unsigned char}, \ldots
          \item boolean types: \alert<3->{\code{bool}}
          \end{itemize}
        \item floating-point types: \code{float}, \alert<3->{\code{double}}, \code{long double}
      \end{itemize}
    \item \code{std::nullptr_t}
      \begin{itemize}
      \item type of the null pointer \code{nullptr}
      \end{itemize}
    \item \code{void}
      \begin{itemize}
      \item denotes absence of type information
      \end{itemize}
  \end{itemize}

  \pause

  Size and representation may depend on the actual computer architecture where
  the program runs

\end{frame}

\begin{frame}[fragile]{\code{int}}

  \begin{columns}
    \begin{column}{.8\textwidth}
      Type representing a signed integer number

      \begin{itemize}
      \item<2-> Set of values: subset of $\mathbb{Z}$
      \item<3-> Operations: addition, subtraction, multiplication, division,
        remainder, comparisons, \ldots
      \item<4-> Representation: 2's complement
      \item<5-> With N bits, values are in the range $[-2^{N-1}, 2^{N-1}-1]$
      \item<6-> Intended to have the natural size suggested by the architecture.
        Typical size is 32 bits (4 bytes)
        \begin{itemize}
        \item $[-2$ $147$ $483$ $648, +2$ $147$ $483$ $647]$
        \end{itemize}
      \item<7-> Default type you should use to count and do integer arithmetic
        \begin{itemize}
        \item \href{https://isocpp.github.io/CppCoreGuidelines/CppCoreGuidelines#es102-use-signed-types-for-arithmetic}{ES.102}
        \end{itemize}
      \end{itemize}
    \end{column}
    \hfill
    \begin{column}{.2\textwidth}
      \uncover<4->{\begin{tabular}{ r | r }
        7 & \alert{0}111 \\
        6 & \alert{0}110 \\
        5 & \alert{0}101 \\
        4 & \alert{0}100 \\
        3 & \alert{0}011 \\
        2 & \alert{0}010 \\
        1 & \alert{0}001 \\
        0 & \alert{0}000 \\
        -1 & \alert{1}111 \\
        -2 & \alert{1}110 \\
        -3 & \alert{1}101 \\
        -4 & \alert{1}100 \\
        -5 & \alert{1}011 \\
        -6 & \alert{1}010 \\
        -7 & \alert{1}001 \\
        -8 & \alert{1}000
      \end{tabular}}
    \end{column}
  \end{columns}

\end{frame}

% the fragile here is needed because of the use of a # character
% https://tex.stackexchange.com/questions/420448/error-illegal-parameter-number-in-definition-of-iterate
\begin{frame}[fragile]{Identifiers}

  \begin{itemize}
  \item An \textit{identifier} is a sequence of letters (including \code{_}) and
    digits, starting with a letter
    \begin{itemize}
    \item Avoid \code{_} at the beginning
    \end{itemize} 

  \item Identifiers are used to name entities in a program
    \begin{itemize}
    \item Choose meaningful names
    \item Get used to write in English
    \end{itemize}
  \item \href{https://isocpp.github.io/CppCoreGuidelines/CppCoreGuidelines#nl8-use-a-consistent-naming-style}{NL.8}
  \end{itemize}

\end{frame}

\begin{frame}{Keywords}

  The following identifiers are reserved
  \vskip .5cm

  {
    \setbeamerfont{localfont}{size=\scriptsize,family=\tt}
    \usebeamerfont{localfont}

    \begin{tabular}{l l l l l}
      alignas          & consteval        & final            & or_eq            & throw            \\
      alignof          & constexpr        & float            & override         & true             \\
      and              & constinit        & for              & private          & try              \\
      and_eq           & const_cast       & friend           & protected        & typedef          \\
      asm              & continue         & goto             & public           & typeid           \\
      auto             & co_await         & if               & register         & typename         \\
      bitand           & co_return        & import           & reinterpret_cast & union            \\
      bitor            & co_yield         & inline           & requires         & unsigned         \\
      bool             & decltype         & int              & return           & using            \\
      break            & default          & long             & short            & virtual          \\
      case             & delete           & module           & signed           & void             \\
      catch            & do               & mutable          & sizeof           & volatile         \\
      char             & double           & namespace        & static           & wchar_t          \\
      char8_t          & dynamic_cast     & new              & static_assert    & while            \\
      char16_t         & else             & noexcept         & static_cast      & xor              \\
      char32_t         & enum             & not              & struct           & xor_eq           \\
      class            & explicit         & not_eq           & switch                              \\
      compl            & export           & nullptr          & template                            \\
      concept          & extern           & operator         & this                                \\
      const            & false            & or               & thread_local                        \\
    \end{tabular}
  }
\end{frame}

\begin{frame}[fragile]{Variables}

  \begin{itemize}
  \item A variable is an identifier that gives a \alert{name} to an
    \textit{object}
  \end{itemize}

  \begin{codeblock}<2->{
\uncover<2->{\tikzref{l1}int i;            // declaration}\uncover<3->{; the value is undefined}
\uncover<4->{\tikzref{l2}i = 4321;         // assignment of a constant}
\uncover<5->{\tikzref{l3}int j\{1234\};      // declaration and initialization in one step}
\uncover<6->{\tikzref{l4}i = j;            // assignment of j\textquotesingle{}s value to i}}\end{codeblock}

  \only<trans>{
    \tikz[overlay] \node[circle,draw,inner sep=1pt] (m1) at ([xshift=-2ex]l1) {\tiny 1};
    \tikz[overlay] \node[circle,draw,inner sep=1pt] (m2) at ([xshift=-2ex]l2) {\tiny 2};
    \tikz[overlay] \node[circle,draw,inner sep=1pt] (m3) at ([xshift=-2ex]l3) {\tiny 3};
    \tikz[overlay] \node[circle,draw,inner sep=1pt] (m4) at ([xshift=-2ex]l4) {\tiny 4};
  }

  \only<beamer>{
    \begin{tikzpicture}[
      mem/.style={
        node font=\ttfamily\scriptsize,
        minimum height=.5cm,
      },
      location/.style={
        mem,
        draw=black!50,
        minimum width=1.2cm,
        fill=green!20!white,
      },
      every node/.style={
        mem,
      },
      anchor=south west,
      node distance=0,
      ]
      \clip (0.1,-1) rectangle (\textwidth-0.1cm,1.5);
      %% \visible<3->{\draw[step=.5cm,gray,thin,dashed] (0, .1cm) grid (\textwidth,0.4cm);}
      \visible<2->{\node at (0,0) [mem,draw=black!50,minimum width=\textwidth,
        "Memory" above] {};}
      \visible<2->{\node at (2,0) [location, "i" below] {
          \alt<-2>{}{\alt<3>{?}{\alt<4-5>{4321}{1234}}}};}
      \visible<5->{\node at (7,0) [location, "j" below] {1234};}
    \end{tikzpicture}
  }

  \only<trans>{
    \vskip -.2 cm

    \begin{tikzpicture}[
      mem/.style={
        node font=\ttfamily\scriptsize,
        minimum height=.5cm,
      },
      location/.style={
        mem,
        draw=black!50,
        minimum width=1.2cm,
        fill=green!20!white,
      },
      every node/.style={
        mem,
      },
      anchor=north west,
      node distance=0,
      ]
      \clip (0.1,.5) rectangle (\textwidth-0.1cm,-3.6);
      \node (mem1) at (0,0) [mem,draw=black!50,minimum width=\textwidth, "Memory" above] {};
      \node at (2,0) [location, "i" above] {?};
      \node (mem2) at (0,-1) [mem,draw=black!50,minimum width=\textwidth] {};
      \node at (2,-1) [location] {4321};
      \node (mem3) at (0,-2) [mem,draw=black!50,minimum width=\textwidth] {};
      \node at (2,-2) [location] {4321};
      \node at (7,0) [phantom word, "j" above] {};
      \node at (7,-2) [location] {1234};
      \node (mem4) at (0,-3) [mem,draw=black!50,minimum width=\textwidth] {};
      \node at (2,-3) [location] {1234};
      \node at (7,-3) [location] {1234};
    \end{tikzpicture}

    \tikz[overlay] \node[circle,draw,inner sep=1pt] at ([xshift=-1ex]mem1.west) {\tiny 1};
    \tikz[overlay] \node[circle,draw,inner sep=1pt] at ([xshift=-1ex]mem2.west) {\tiny 2};
    \tikz[overlay] \node[circle,draw,inner sep=1pt] at ([xshift=-1ex]mem3.west) {\tiny 3};
    \tikz[overlay] \node[circle,draw,inner sep=1pt] at ([xshift=-1ex]mem4.west) {\tiny 4};

    \vskip -.2 cm
  }

  \uncover<7->{NB At the end \code{i} and \code{j} have the same value but remain
  \alert{distinct} objects}

\end{frame}

\begin{frame}{Literals}

  A literal is a constant \textbf{value} of a certain \textbf{type} included in
  the source code

  \pause

  \begin{itemize}
  \item integer
  \item floating point
  \item character
  \item string
  \item boolean
  \item null pointer
  \item user-defined
  \end{itemize}

\end{frame}

\begin{frame}{Integer literals}

  \begin{description}[<+->]
  \item[decimal] non-\code{0} decimal digit followed by zero or more digits
    \begin{itemize}[<.->]
    \item \code{1 -98 123456789 -1\textquotesingle{}234\textquotesingle{}567\textquotesingle{}890}
    \end{itemize}
  \item[binary] \code{0b} or \code{0B} followed by binary digits
    \begin{itemize}[<.->]
    \item \code{0b1101111010101101 0B111\textquotesingle{}0101\textquotesingle{}1011\textquotesingle{}1100\textquotesingle{}1101\textquotesingle{}0001\textquotesingle{}0101}
    \end{itemize}
  \item[exadecimal] \code{0x} or \code{0X} followed by hexadecimal digits
    \begin{itemize}[<.->]
    \item \code{-0xdead 0xDEad123f 0XdeAD\textquotesingle{}123F}
    \end{itemize}
  \item[octal] \code{0} followed by octal digits
    \begin{itemize}[<.->]
    \item \code{01 -077 07\textquotesingle{}654\textquotesingle{}321}
    \item N.B. A \code{0} in front of a number is meaningful!
    \end{itemize}
  \end{description}

  \uncover<+->{Integer literals are of type \code{int}}

\end{frame}

\begin{frame}[fragile]{\code{std::string}}

  \begin{itemize}[<+->]

  \item A \textit{compound} (\textit{user-defined}) type to represent a
    string of characters
  \item Provided by the C++ Standard Library
  \item Many operations available
  \item An \code{std::string} can be initialized with a string literal, a
    sequence of escaped or non-escaped characters between double quotes
    \begin{itemize}[<.->]
    \item \code{"hello" "hello\bslashn{}world" "hello
        \bslash{"}world\bslash{"}"}
    \item \code{\bslash{}n} means ``newline''
    \end{itemize}
  \item The type of a string literal is \textbf{not} \code{std::string}
  \end{itemize}

  \begin{codeblock}<+->{
std::string corso\{"Programmazione per la Fisica"\};
\uncover<+->{corso = corso \tikzref{strplusref}+ "\bslash{n}Anno Accademico 2023/2024";}}\end{codeblock}

\uncover<+->{
  \begin{tikzpicture}[remember picture,overlay]
    \node[draw=red,line width=1pt,opacity=.7,rounded corners, minimum
    width=1.5ex] at ([xshift=.5ex]strplusref) (strplusbox) {};
    \node[below=of strplusbox,draw=red,line width=1pt,opacity=.7,rounded
    corners] {concatenate} edge[->,line width=1pt,red,opacity=.7] (strplusbox)
    {};
  \end{tikzpicture}
}

\end{frame}

\begin{frame}[fragile]{Expressions}

  \begin{itemize}
  \item An expression is a sequence of operators and their operands
    that specifies a computation
  \item Literals and variables are typical operands, but there are others

  \begin{codeblock}<2->{
\uncover<2->{1 + 2}
\uncover<3->{\alert<8->{i = 1 + 2}       // assignment}
\uncover<4->{i == j          // equality comparison}
\uncover<5->{sqrt(x) > 1.42}
\uncover<6->{\alert<8->{std::cout <{}< "hello, " <{}< name <{}< \upquote{\bslash{}n}}}
\uncover<7->{a > 0 ? a - 1 : a + 1}}\end{codeblock}

\item<8-> The evaluation of an expression typically produces a result
\item<9-> Some expressions have \textit{side-effects}
  \begin{itemize}
  \item They modify the state of the program, i.e. the state of memory, or the
    external world
  \end{itemize}
\item<10-> Avoid complicated expressions
  (\href{https://isocpp.github.io/CppCoreGuidelines/CppCoreGuidelines#es40-avoid-complicated-expressions}{ES.40})
\end{itemize}

\end{frame}

\begin{frame}[fragile]{Operators}

  \begin{columns}[t]

    \column{0.12\textwidth}
    {\scriptsize Arithmetic}
    \begin{codeblock}
+a
-a
a + b
a - b
a * b
a / b
a % b
~a
a & b
a | b
a ^ b
a {<}< b
a {>}> b\end{codeblock}

    \column{0.12\textwidth}
    \centering{\scriptsize Logical}
    \begin{codeblock}
!a
a && b
a || b\end{codeblock}

    \column{0.13\textwidth}
    \centering{\scriptsize Comparison}
    \begin{codeblock}
a == b
a != b
a {<} b
a {>} b
a {<}= b
a >= b\end{codeblock}

    \column{0.13\textwidth}
    \centering{\scriptsize Increment}
    \begin{codeblock}
++a
{-}-a
a++
a{-}-\end{codeblock}

    \column{0.14\textwidth}
    \centering{\scriptsize Assignments}
    \begin{codeblock}
a = b
a += b
a -= b
a *= b
a /= b
a %= b
a &= b
a |= b
a ^= b
a <{}<= b
a >{}>= b\end{codeblock}

    \column{0.09\textwidth}
    \centering{\scriptsize Access}
    \begin{codeblock}
a[b]
*a
&a
a->b
a.b\end{codeblock}

    \column{0.09\textwidth}
    \centering{\scriptsize Other}
    \begin{codeblock}
a(\ddd)
a, b
? :\end{codeblock}

  \end{columns}

  \vskip .5cm

  Plus: casts, allocation and deallocation, static introspection, \ldots

  \begin{itemize}[<+->]
  \item Rules exist for associativity, commutativity and precedence
    \begin{itemize}[<.->]
    \item When in doubt use parentheses
      (\href{https://isocpp.github.io/CppCoreGuidelines/CppCoreGuidelines#es41-if-in-doubt-about-operator-precedence-parenthesize}{ES.41})
    \end{itemize}
  \item Many operators can be \alert{overloaded} for user-defined types
    \begin{itemize}[<.->]
    \item e.g. \code{+} to concatenate strings
    \end{itemize}
  \end{itemize}

\end{frame}
