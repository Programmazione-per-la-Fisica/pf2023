% !TeX root = ./pf2023.tex

%\includeonlyframes{current}

\section*{Flow control}

\begin{frame}{Algorithm}

  A finite sequence of precisely defined steps to solve a problem

\end{frame}

\begin{frame}<1-2|trans:1>[fragile,label=firstexample]{Sum of two numbers}

  A program that reads two numbers from input and writes their sum to output

  \begin{columns}[T]
    \begin{column}{.5\textwidth}
      \centering
  \uncover<2->{
    \begin{tikzpicture}
      [auto,
      startstop/.style = {rectangle, rounded corners=5pt, draw=black, thick, node font=\scriptsize,
        text width=4em, align=flush center},
      inout/.style     = {trapezium, trapezium left angle=60,trapezium right angle=120, thick, draw=green, fill=green!20,
        text width=4em, align=flush center, node font=\scriptsize},
      block/.style     = {rectangle, draw=blue, thick, fill=blue!20,align=center, rounded corners=2pt,
        text width=4em, align=flush center, node font=\scriptsize},
      decision/.style  = {diamond, aspect=1.5, draw=red, thick, fill=red!20, text width=4em, align=flush center, inner sep=1pt, node font=\scriptsize},
      line/.style      = {draw, thick, -latex, shorten >=2pt}]
      \matrix [column sep=5mm,row sep=4mm]
      {
        % row 1
        \node [startstop] (start) {start}; \\
        % row 2
        \node [inout] (in) {read a, b}; \\
        % row 3
        \node [block,text width=7em] (result) {result $\leftarrow$ a + b}; \\
        % row 4
        \node [inout] (out) {write result}; \\
        % row 5
        \node [startstop] (stop) {stop}; \\
      };
      \begin{scope}[every path/.style=line]
        \path (start)  -- (in);
        \path (in)     -- (result);
        \path (result) -- (out);
        \path (out)    -- (stop);
      \end{scope}
    \end{tikzpicture}
  }
    \end{column}
    \begin{column}{.5\textwidth}
      \begin{codeblock}<3-|trans:2>{
#include <iostream>

int main()
\{
  \uncover<4->{\alert<4|trans:0>{int a;
  int b;
  std::cin >> a >> b;}}
  \uncover<5->{\alert<5|trans:0>{int result\{a + b\};}}
  \uncover<6->{\alert<6|trans:0>{std::cout <{}< result <{}< \bslashn;}}
\}}\end{codeblock}

    \end{column}
  \end{columns}

  \uncover<7-|trans:2>{
    \begin{tikzpicture}[
      mem/.style={
        node font=\ttfamily\scriptsize,
        minimum height=.5cm,
      },
      location/.style={
        mem,
        draw=black!50,
        minimum width=1.2cm,
        fill=green!20!white,
      },
      every node/.style={
        mem,
      },
      anchor=south west,
      node distance=0,
      ]
      \clip (0.1,-1) rectangle (\textwidth-0.1cm,1.5);
      \node at (0,0) [mem,draw=black!50,minimum width=\textwidth, "Memory" above] {};
      \node[location] (c) at (2,0) ["std{::}cout" below] {\ddd};
      \node[location] (a) at (5,0) ["a" below] {3};
      \node[location] (b)          [right=of a, "b" below] {4};
      \node[location] (r)          [right=of b, "result" below] {7};
    \end{tikzpicture}
  }

\end{frame}

\begin{frame}{Statement}
  Statements are units of code that are executed in sequence

  \begin{itemize}
  \item expression statement
  \item compound statement or block
  \item declaration statement
  \item selection statement
  \item iteration statement
  \item jump statement
  \item \ldots
  \end{itemize}

\end{frame}

\againframe<3-|trans:2>{firstexample}

\begin{frame}[fragile]{Expression statement}

  \begin{itemize}[<+->]
  \item An expression followed by a semicolon (\code{;})
  \item The value of the expression (if any) is discarded
  \item The expression can have side effects
    \begin{itemize}[<.->]
    \item Indeed, we often write statements with the purpose to change data in
      memory or to do I/O (input/output)
    \end{itemize}
  \end{itemize}

  \begin{codeblock}
\uncover<1->{b + 2;}
\uncover<3->{a = b + 2;
std::cin >> a >> b;}
\uncover<4->{; // empty statement}\end{codeblock}

\end{frame}

\begin{frame}[fragile]{Compound statement (or block)}

  A sequence of zero or more statements enclosed between braces
  (\code{\{\}})

  \begin{codeblock}
\{
   found = true;
   ++i;
   int n\{3\};
   std::cout <{}< n;
\}\end{codeblock}

  \begin{codeblock}
\{ found = true; ++i; \ddd \} // all on one line\end{codeblock}

\end{frame}

\begin{frame}[fragile]{Declaration statement}

  \begin{itemize}[<+->]
  \item A declaration statement introduces one or more new identifiers into a
    \Cpp{} program, possibly initializing them
    \begin{itemize}
    \item typically variables, but not only
    \end{itemize}

  \item A declaration of a \textbf{variable} in a \textbf{block} makes the
    variable of \alert{automatic storage duration}, unless otherwise specified
    \begin{itemize}
    \item the corresponding object is automatically created each time the
      declaration is executed
    \item<.-> the corresponding object is automatically destroyed each time the
      execution reaches the end of the block
    \end{itemize}

  \item A declaration should introduce only one identifier
    (\href{https://isocpp.github.io/CppCoreGuidelines/CppCoreGuidelines#es10-declare-one-name-only-per-declaration}{ES.10})

  \item A variable should be declared only in the moment it's actually needed
    (\href{https://isocpp.github.io/CppCoreGuidelines/CppCoreGuidelines#es21-dont-introduce-a-variable-or-constant-before-you-need-to-use-it}{ES.21})

  \item A variable should be initialized at the point of declaration
    (\href{https://isocpp.github.io/CppCoreGuidelines/CppCoreGuidelines#es20-always-initialize-an-object}{ES.20})
    \begin{itemize}
    \item There are very few exceptions, if any, to this recommendation
    \end{itemize}
  \end{itemize}

\end{frame}

\begin{frame}{Scope}

  The scope of a name appearing in a program is the, possibly
  discontiguous, portion of source code where that name is valid

  \begin{itemize}
  \item block scope
  \item function scope
  \item class scope
  \item namespace scope (including the global one)
  \item \ldots
  \end{itemize}

\end{frame}

\begin{frame}[fragile]{Block scope}

  \begin{itemize}[<+->]
  \item The scope of a name declared in a block starts at the point of
    declaration and ends at the \code{\}}

    \begin{codeblock}
\{
  num + 1;         // error
  int num;         // scope of num begins only here
  std::cin >> num; // ok
\}                  // scope of num ends here
std::cout <{}< num;  // error\end{codeblock}

  \item<2-> The scope of a variable should be as small as possible
   (\href{https://isocpp.github.io/CppCoreGuidelines/CppCoreGuidelines#es5-keep-scopes-small}{ES.5})
    \begin{itemize}
    \item Declare a variable only when it's needed
    \item If possible/meaningful, initialize it
    \end{itemize}
  \end{itemize}

\end{frame}

\begin{frame}[fragile]{The smallest of two numbers}

  A program that reads two numbers from input and writes the smallest to output

  \begin{columns}[T]
    \begin{column}{.5\textwidth}
  \uncover<2->{
    \begin{tikzpicture}
      [auto,
      startstop/.style = {rectangle, rounded corners=5pt, draw=black, thick, node font=\scriptsize,
        text width=3em, align=flush center},
      inout/.style     = {trapezium, trapezium left angle=60,trapezium right angle=120, thick, draw=green, fill=green!20,
        text width=4em, align=flush center, node font=\scriptsize},
      block/.style     = {rectangle, draw=blue, thick, fill=blue!20,align=center, rounded corners=2pt,
        text width=5em, align=flush center, node font=\scriptsize},
      decision/.style  = {diamond, aspect=1.5, draw=red, thick, fill=red!20, text width=4em, align=flush center, inner sep=1pt, node font=\scriptsize},
      line/.style      = {draw, thick, -latex, shorten >=2pt}]
      \matrix [column sep=0mm,row sep=4mm,ampersand replacement=\&]
      {
        % row
        \& \node [startstop] (start) {start}; \\
        % row
        \& \node [inout] (in) {read a, b}; \\
        % row
        \& \node [decision] (less) {a < b}; \\[-4mm]
        % row
        \node [block] (a) {result $\leftarrow$ a}; \&
        \& \node [block] (b) {result $\leftarrow$ b}; \\
        % row
        \& \coordinate (join); \\
        % row
        \& \node [inout] (out) {write result}; \\
        % row
        \& \node [startstop] (stop) {stop}; \\
      };
      \begin{scope}[every path/.style=line]
        \path (start)  -- (in);
        \path (in)     -- (less);
        \path (less)   -| node [near start,above] {\scriptsize{yes}} (a);
        \path (less)   -| node [near start,above] {\scriptsize{no}} (b);
        \path[-,shorten >=0pt] (a)   |- (join);
        \path[-,shorten >=0pt] (b)   |- (join);
        \path (join) -- (out);
        \path (out)    -- (stop);
      \end{scope}
    \end{tikzpicture}
  }
    \end{column}
    \begin{column}{.5\textwidth}
      \begin{codeblock}<3->{
#include <iostream>

int main()
\{
  \uncover<4->{\alert<4|trans:0>{int a;
  int b;
  std::cin >> a >> b;}}
  \uncover<7->{\alert<7|trans:0>{int result;}}
  \uncover<5->{\alert<5|trans:0>{if (a < b) \{}
    \uncover<8->{\alert<8|trans:0>{result = a;} // (1)}
  \alert<5|trans:0>{\} else \{}
    \uncover<8->{\alert<8|trans:0>{result = b;} // (2)}
  \alert<5|trans:0>{\}}}
  \uncover<6->{\alert<6|trans:0>{std::cout << result << \bslashn;} // (3)}
\}}\end{codeblock}

    \end{column}
  \end{columns}

  \uncover<9->{\code{result} cannot be defined in (1) and/or (2), otherwise it
    would not be visible in (3)}

\end{frame}

\begin{frame}[fragile]{\code{if then else}}

  \begin{itemize}
  \item<1-> Selection statement to choose one of two flows of instructions
    depending on a boolean condition
  \item<2-> Two (basic) forms:
    \begin{itemize}
    \item<3-> \code{if (} \textit{condition-expr} \code{)} \textit{statement} \code{else} \textit{statement}
    \item<4-> \code{if (} \textit{condition-expr} \code{)} \textit{statement}
    \end{itemize}

  \begin{columns}[t]

    \begin{column}{.45\textwidth}

      \begin{codeblock}<3->{
\alert{if} (a < b) \{
  result = a;   // "true" branch
\} \alert{else} \{
  result = b;   // "false" branch
\}}\end{codeblock}

    \end{column}

    \begin{column}{.45\textwidth}
      \begin{codeblock}<4->{
int i\{\ddd\};
\alert{if} (i < 0) \{
  i = -i;   // "true" branch
\}}\end{codeblock}

    \end{column}

  \end{columns}

  \item<5-> \textit{condition-expr} can be any expression whose result is of type
    (convertible to) \code{bool}, i.e. either true or false

  \item<6-> \textit{statement} can be a block, possibly including multiple
    statements
    \begin{itemize}
    \item In fact, the statement should always be a block
    \end{itemize}
  \end{itemize}

\end{frame}

\begin{frame}[fragile]{\code{bool}}

  Type representing a boolean value
  \begin{itemize}
  \item Set of values: ${false, true}$
  \item Operations: logical conjunction (\textit{and}), disjunction
    (\textit{or}) and negation (\textit{not}), assignment, comparison
  \item Size: typically $1$ byte
  \item Representation: like the \code{int}s \code{0} and \code{1}
  \item Literals: \code{false} and \code{true}

    \begin{codeblock}<2->{
\uncover<2->{bool b\{true\};}
\uncover<3->{b \tikzref{aref}= i =\tikzref{eref}= j;}
\uncover<6->{bool b2\{i !\tikzref{neref}= 1234;\}}
\uncover<7->{bool b3\{b2 \alert<7>{\&\&} i < j\};    // and
bool b4\{i < j \alert<7>{||} i < k\}; // or
bool b5\{\alert<7>{!}\alert<8>{(}i == j\alert<8>{)\}};      // not}}\end{codeblock}

  \item<8-> Note the use of parenthesis in \code{!(i == j)} to have the right
    precedence of application of operators
  \end{itemize}

\uncover<4->{
  \begin{tikzpicture}[remember picture,overlay]
    \uncover<4-5>{
      \node[draw=red,line width=1pt,opacity=.7,rounded corners, minimum width=1.5ex] at ([xshift=.5ex]aref) (abox){};
      \node (atext) at (current page.south west) [anchor=south west,draw=red,line width=1pt,opacity=.7,rounded corners] {assignment} edge[->,line width=1pt,red,opacity=.3] (abox);
    }
    \uncover<5>{
      \node[draw=red,line width=1pt,opacity=.7,rounded corners,minimum width=3ex] at (eref) (ebox){};
      \node (etext) [right=of atext,draw=red,line width=1pt,opacity=.7,rounded corners] {equality} edge[->,line width=1pt,red,opacity=.3] (ebox);
    }
    \uncover<6>{
      \node[draw=red,line width=1pt,opacity=.7,rounded corners,minimum width=3ex] at (neref) (nebox){};
      \node (netext) [right=of etext,draw=red,line width=1pt,opacity=.7,rounded corners] {inequality} edge[->,line width=1pt,red,opacity=.3] (nebox);
    }
  \end{tikzpicture}
}

\end{frame}

\begin{frame}{Logical operations}

  \begin{description}[<+->]

  \item[and] True if both operands are true
    \vskip .5em
    {\scriptsize\begin{tabular}{ c c | c }
      op1 & op2 & op1 \textbf{\&\&} op2 \\
      \hline
      \code{false} & \code{false} & \code{false} \\
      \code{false} & \code{true} & \code{false} \\
      \code{true} & \code{false} & \code{false} \\
      \code{true} & \code{true} & \code{true} \\
    \end{tabular}}
    \vskip .5em

    \code{op2} is evaluated only if \code{op1} is \code{true}

  \item[or] True if at least one operand is true
    \vskip .5em

    {\scriptsize\begin{tabular}{ c c | c }
      op1 & op2 & op1 \textbf{||} op2 \\
      \hline
      \code{false} & \code{false} & \code{false} \\
      \code{false} & \code{true} & \code{true} \\
      \code{true} & \code{false} & \code{true} \\
      \code{true} & \code{true} & \code{true} \\
    \end{tabular}}
    \vskip .5em

    \code{op2} is evaluated only if \code{op1} is \code{false}

  \item[not] Operand's negation (or logical complement)
    \vskip .5em

    {\scriptsize\begin{tabular}{ c | c }
      op & \textbf{!}op \\
      \hline
      \code{false} & \code{true} \\
      \code{true}  & \code{false} \\
    \end{tabular}}
    \vskip .5em

  \end{description}

\end{frame}

\begin{frame}[fragile]{Example: is a number even?}

  A program that reads a number from input and tells if it's even

  \begin{columns}[T]
    \begin{column}{.6\textwidth}
  \uncover<2->{
    \begin{tikzpicture}
      [auto,
      startstop/.style = {rectangle, rounded corners=5pt, draw=black, thick, node font=\scriptsize,
        text width=3em, align=flush center},
      inout/.style     = {trapezium, trapezium left angle=60,trapezium right angle=120, thick, draw=green, fill=green!20,
        text width=4em, align=flush center, node font=\scriptsize},
      block/.style     = {rectangle, draw=blue, thick, fill=blue!20,align=center, rounded corners=2pt,
        text width=5em, align=flush center, node font=\scriptsize},
      decision/.style  = {diamond, aspect=1.5, draw=red, thick, fill=red!20, text width=4em, align=flush center, inner sep=1pt, node font=\scriptsize},
      line/.style      = {draw, thick, -latex, shorten >=2pt}]
      \matrix [column sep=0mm,row sep=4mm,ampersand replacement=\&]
      {
        % row
        \& \node [startstop] (start) {start}; \\
        % row
        \& \node [inout] (in) {read num}; \\
        % row
        \& \node [block,text width=7em] (mod) {r $\leftarrow$ num \textit{mod} 2}; \\
        % row
        \& \node [decision] (iszero) {r == 0}; \\[-4mm]
        % row
        \node [block,text width=6em] (even) {result $\leftarrow$ true}; \&
        \& \node [block,text width=6em] (odd) {result $\leftarrow$ false}; \\
        % row
        \& \coordinate (join); \\
        % row
        \& \node [inout,text width=6em] (out) {write result}; \\
        % row
        \& \node [startstop] (stop) {stop}; \\
      };
      \begin{scope}[every path/.style=line]
        \path (start)  -- (in);
        \path (in)     -- (mod);
        \path (mod)    -- (iszero);
        \path (iszero) -| node [near start,above] {\scriptsize{yes}} (even);
        \path (iszero) -| node [near start,above] {\scriptsize{no}} (odd);
        \path[-,shorten >=0pt] (even) |- (join);
        \path[-,shorten >=0pt] (odd)  |- (join);
        \path (join) -- (out);
        \path (out)    -- (stop);
      \end{scope}
    \end{tikzpicture}
  }
    \end{column}

    \begin{column}{.4\textwidth}
      \begin{codeblock}<3->{
#include <iostream>

int main()
\{
\uncover<4->{\alert<4|trans:0>{  int num;
  std::cin >{}> num;}}
\uncover<5->{\alert<5|trans:0>{  int r\{num \% 2\};}}
\uncover<7->{\alert<7|trans:0>{  bool result;}}
\uncover<6->{\alert<6|trans:0>{  if (r == 0) \{}}
\uncover<7->{\alert<7|trans:0>{    result = true;}}
\uncover<6->{\alert<6|trans:0>{  \} else \{}}
\uncover<7->{\alert<7|trans:0>{    result = false;}}
\uncover<6->{\alert<6|trans:0>{  \}}}
\uncover<9->{  // print 0/1}
\uncover<8->{\alert<8|trans:0>{  std::cout << result << \bslashn;}}
\uncover<9->{  // print false/true
  std::cout <{}< \alert<9|trans:0>{std::boolalpha}
      <{}< result <{}< \bslashn;}
\}}\end{codeblock}
    \end{column}

  \end{columns}

  \uncover<11->{The whole \code{if} can be removed defining \code{bool result\{r == 0\};}}

\end{frame}

\begin{frame}{Exercise: the smallest of three numbers}

  Write a program that reads three numbers from standard input and writes the
  smallest one to standard output

\end{frame}

\begin{frame}[fragile]{Integer square root}

  Write a program that computes the integer square root of a non-negative
  integer number, i.e. the largest integer number whose square is not greater
  than the given number

  \begin{columns}[T]
    \begin{column}{.5\textwidth}
      \uncover<2->{\begin{tikzpicture}
          [auto,
          startstop/.style = {rectangle, rounded corners=5pt, draw=black, thick, node font=\scriptsize,
            text width=3em, align=flush center},
          inout/.style     = {trapezium, trapezium left angle=60,trapezium right angle=120, thick, draw=green, fill=green!20,
            text width=4em, align=flush center, node font=\scriptsize},
          block/.style     = {rectangle, draw=blue, thick, fill=blue!20,align=center, rounded corners=2pt,
            text width=5em, align=flush center, node font=\scriptsize},
          decision/.style  = {diamond, aspect=1.5, draw=red, thick, fill=red!20, text width=4em, align=flush center, inner sep=1pt, node font=\scriptsize},
          line/.style      = {draw, thick, -Classical TikZ Rightarrow, shorten >=1pt}]
          \matrix [column sep=4mm,row sep=3mm,ampersand replacement=\&]
          {
            % row
            \& \node [startstop] (start) {start}; \\
            % row
            \& \node [inout] (in) {read $n$}; \\
            % row
            \& \node [block] (initi) {$i \gets 1$}; \\[-1mm]
            % row
            \& \coordinate (beforecond1); \\
            % row
            \node [block] (inc) {$i \gets i + 1$}; \& \node [decision] (cond1) {$i \times i < n$}; \\
            % row
            \node [block] (dec) {$i \gets i - 1$};
            \& \node [decision] (cond2) {$i \times i > n$}; \\[-1mm]
            % row
            \& \coordinate (aftercond2); \\
            % row
            \& \node [inout] (out) {write $i$}; \\
            % row
            \& \node [startstop] (stop) {stop}; \\
          };
          \begin{scope}[every path/.style=line]
            \path (start)  -- (in);
            \path (in)     -- (initi);
            \path (initi)  -- (cond1);
            \path (cond1)  -- node {\scriptsize{no}} (cond2);
            \path (cond1)  -- node {\scriptsize{yes}} (inc);
            \path (inc)    |- (beforecond1);
            \path (cond2)  -- node [near start] {\scriptsize{no}} (out);
            \path (cond2)  -- node {\scriptsize{yes}} (dec);
            \path (dec)    |- (aftercond2);
            \path (out)    -- (stop);
          \end{scope}
        \end{tikzpicture}}

    \end{column}

    \begin{column}{.5\textwidth}
      \begin{codeblock}<3->{
#include <iostream>

int main()
\{
  int n;
  std::cin >> n;
  int i\{1\};
  while (i * i < n) \{
    ++i;   // equivalent to i = i + 1
  \}
  if (i * i > n) \{
    --i;   // equivalent to i = i - 1
  \}
  std::cout <{}< i <{}< \bslashn;
\}}\end{codeblock}
    \end{column}

  \end{columns}

\end{frame}

\begin{frame}[fragile]{\code{while} loop}

  \begin{center}
    \code{while (} \textit{condition-expr} \code{)} \textit{statement}
  \end{center}

  \pause

  \begin{itemize}[<+->]
  \item Execute repeatedly \textit{statement} until \textit{condition-expr}
    becomes false
  \item \textit{condition-expr} is evaluated at the beginning of each iteration
    \begin{itemize}
    \item If \textit{condition-expr} is already \code{false} at the beginning,
      \textit{statement} is never executed
    \end{itemize}
  \item \textit{statement} can be any statement
    \begin{itemize}
    \item It's better if the statement is always a block
    \end{itemize}
  \item \textit{statement} should modify something so that the evaluation of
    \textit{condition-expr} may change
    \begin{itemize}[<.->]
    \item Otherwise the loop may never terminate
    \end{itemize}
  \end{itemize}

\end{frame}

\begin{frame}[fragile]{Sum of the first N numbers}
  Write a program that reads a non-negative integer number N from standard
  input, computes the sum of the first N numbers and writes the result to
  standard output

  \begin{columns}[T]

    \begin{column}{.5\textwidth}
      \begin{codeblock}<2->{
#include <iostream>

int main()
\{
  int n;
  std::cin >> n;
  int sum\{0\}; // or just sum\{\}
  \alert<3->{int i\{1\};}
  while (\alert<3->{i <= n}) \{
    sum += i;   // sum = sum + i
    \alert<3->{++i};
  \}
  std::cout <{}< sum <{}< \bslashn;
\}}\end{codeblock}

    \end{column}

    \begin{column}{.5\textwidth}
      \begin{codeblock}<4->{
#include <iostream>

int main()
\{
  int n;
  std::cin >> n;
  int sum\{0\};

  \textbf<4>{for} (\alert<4>{int i\{1\}}; \alert<4>{i <= n}; \alert<4>{++i}) \{
    sum += i;

  \}
  std::cout <{}< sum <{}< \bslashn;
\}}\end{codeblock}
    \end{column}

  \end{columns}
\end{frame}

\begin{frame}[fragile]{\code{for} loop}

  \code{for (} \textit{init-statement} \opt{\textit{condition-expr}} \code{;}
  \opt{\textit{expression}} \code{)} \textit{statement}

  \pause

  \begin{itemize}[<+->]
  \item Execute \textit{init-statement}, which may be a single \code{;}
    \begin{itemize}
    \item If \textit{init-statement} contains declarations (e.g. the \code{i}
      variable in the previous example), the scope of the declared names is
      the loop
    \end{itemize}
  \item Execute repeatedly \textit{statement} until \textit{condition-expr} becomes false
  \item \textit{condition-expr} is evaluated at the beginning of each iteration
    \begin{itemize}
    \item If \textit{condition-expr} is already false at the beginning,
      \textit{statement} is never executed
    \item But \textit{init-statement} is always executed
    \end{itemize}
  \end{itemize}

\end{frame}

\begin{frame}[fragile]{\code{for} loop \insertcontinuationtext}

  \code{for (} \textit{init-statement} \opt{\textit{condition-expr}} \code{;}
  \opt{\textit{expression}} \code{)} \textit{statement}

  \pause

  \begin{itemize}[<+->]
  \item \textit{expression} is evaluated at the end of each iteration
  \item \textit{statement} and/or \textit{expression} should modify something so
    that the evaluation of \textit{condition-expr} may change
    \begin{itemize}[<.->]
    \item Otherwise the loop may never terminate
    \end{itemize}
  \item A \code{for} loop can always be transformed into a \code{while} loop and
    viceversa
    \begin{itemize}
    \item Prefer a \code{for} loop when there is an obvious variable that
      controls the loop
    \item<.->
      \href{https://isocpp.github.io/CppCoreGuidelines/CppCoreGuidelines#es72-prefer-a-for-statement-to-a-while-statement-when-there-is-an-obvious-loop-variable}{ES.72},
      \href{https://isocpp.github.io/CppCoreGuidelines/CppCoreGuidelines#es73-prefer-a-while-statement-to-a-for-statement-when-there-is-no-obvious-loop-variable}{ES.73}
    \end{itemize}
  \end{itemize}
\end{frame}

\begin{frame}[fragile]{Integer square root with a \code{for} loop}

  Write a program that computes the integer square root of a non-negative
  integer number, i.e. the largest integer number whose square is not greater
  than the given number

  \begin{columns}[T]
    \begin{column}{.35\textwidth}
      \begin{codeblock}
#include <iostream>

int main()
\{
  int n;
  std::cin >> n;
  int i\{1\};
  while (i * i < n) \{
    ++i;
  \}
  if (i * i > n) \{
    --i;
  \}
  std::cout <{}< i <{}< \bslashn;
\}\end{codeblock}
    \end{column}

    \begin{column}{.65\textwidth}
      \begin{codeblock}<2->{
#include <iostream>

int main()
\{
  int n;
  std::cin >> n;
  int i\{1\};  // NB outside the loop
  for ( ; i * i < n; ++i) ;  // or \{\}


  if (i * i > n) \{
    --i;
  \}
  std::cout << i << \bslashn;
\}}\end{codeblock}
    \end{column}

  \end{columns}

\end{frame}

\begin{frame}{Exercise: the largest/smallest of N numbers}

  Write a program that reads an arbitrary sequence of numbers from standard
  input and writes the largest (or smallest) one to standard output

  Hint 1: press Ctrl-D in the terminal to tell the program there is nothing more
  to read from standard input

  Hint 2: the expression \code{std::cin.good()} tells if it's still possible to
  read something from \code{std::cin}
\end{frame}

\begin{frame}[fragile]{\code{double}}

  Type representing a floating-point number

  \begin{itemize}
  \item<2-> Set of values: subset of $\mathbb{R}$
  \item<3-> Operations: addition, subtraction, multiplication, division, comparisons, \ldots
  \item<4-> Representation: \href{https://en.wikipedia.org/wiki/IEEE_754}{IEEE 754}
  \item<5-> 64 bits, smallest values $\approx\pm10^{-308}$, largest values $\approx\pm10^{308}$
  \item<6-> Precision is about 16 decimal digits
  \item<6-> Literals in the form \opt{\textit{sign}} \textit{significand} \opt{\textit{exponent}}
    \begin{itemize}
    \item \code{42.0 1. -1.5 12.34e3 -.34E-3 -1234e-2}
    \item \code{\ddd{}e\textit{n}}/\code{\ddd{}E\textit{n}} means $\times 10^{n}$
    \end{itemize}
  \end{itemize}
\end{frame}

\begin{frame}[fragile]{\code{float}}

  Type representing a floating-point number

  \begin{itemize}
  \item<2-> Set of values: subset of $\mathbb{R}$
  \item<3-> Operations: addition, subtraction, multiplication, division, comparisons, \ldots
  \item<4-> Representation: \href{https://en.wikipedia.org/wiki/IEEE_754}{IEEE 754}
  \item<5-> 32 bits, smallest values $\approx\pm10^{-38}$, largest values $\approx\pm10^{38}$
  \item<6-> Precision is about 7 decimal digits
  \item<6-> Literals in the form \opt{\textit{sign}} \textit{significand} \opt{\textit{exponent}} \code{F}
    \begin{itemize}
    \item \code{42.0f 1.F -1.5f 12.34e3F -.34E-3f -1234e-2F}
    \item Same as double but with an \code{f} or \code{F} suffix
    \end{itemize}
  \end{itemize}
\end{frame}

\begin{frame}[fragile]{Handle floating-point numbers with care}

  \begin{itemize}
  \item<1-> Floating-point numbers are \textbf{not} real numbers
    \begin{itemize}
    \item Finite representation $\implies$ there is a previous and a next
    \end{itemize}
  \item<2-> When managing floating-point numbers one should take into account
    rounding errors and inexact representations
    \begin{codeblock}<3->{
float x\{1\textquotesingle000\textquotesingle000\textquotesingle000.f\};
\uncover<4->{std::cout << std::setprecision(10)
  << x - 32 << \upquote{ } << x << \upquote{ } << x + 32}\uncover<5->{ // 1000000000 1000000000 1000000000}
\uncover<6->{  << x + 32 - x << \upquote{ } << x - x + 32;}\uncover<7->{    // 0 32}}\end{codeblock}
  \item<8-> Floating-point math in general is not associative
  \item<9-> When comparing floating-point numbers avoid the use of equality
  \item<10-> See \href{https://cr.yp.to/2005-590/goldberg.pdf}{What Every
      Computer Scientist Should Know About Floating Point Arithmetic}
 \end{itemize}

\end{frame}

\begin{frame}{Exercise: accumulate 0.001}
  Write a program that accumulates the \code{float} \code{0.001} (think a time
  step of one millisecond) 1'000'000 times and prints the accumulated value.
  Discuss the result.
\end{frame}

\begin{frame}[fragile]{Standard mathematical functions}

  The \href{https://en.cppreference.com/w/cpp/header/cmath}{\code{cmath}} header
  includes many ready-to-use mathematical functions
  \begin{itemize}
  \item Exponential
  \item Power
  \item Trigonometric
  \item Interpolation
  \item Hyperbolic
  \item Floating-point manipulation, classification and comparison
  \item \ldots
  \end{itemize}

  \begin{codeblock}<2->{
#include <cmath>
\ddd
double x\{\ddd\};
std::sqrt(x);
std::pow(x, .5);
std::sin(x);
std::log(x);
std::abs(x);}\end{codeblock}
\end{frame}

\begin{frame}[fragile]{Type conversions}

    \begin{itemize}
    \item A value of type \code{T1} may be converted \textbf{implicitly} to a
      value of type \code{T2} in order to match the expected type in a certain
      situation
      \begin{codeblock}<2->{
1 + 2.3}\end{codeblock}
      \begin{itemize}[<3->]
      \item between numbers and bool
      \item between signed and unsigned numbers
      \item between numbers of different size
      \item between integral and floating-point numbers
      \item \ldots
      \end{itemize}

    \item<4-> Conversions sometimes are surprising
    \item<5-> Conversions can be explicit using \code{static_cast}
      \begin{codeblock}<.->{
1 + static_cast<int>(2.3)}\end{codeblock}
    \item<6-> Mechanims exist to define implicit and explicit conversions
      involving user-defined types

    \end{itemize}

\end{frame}

\begin{frame}[fragile]{\code{const}-safety}

  \begin{itemize}
  \item Data qualified as \code{const} is logically immutable
  \item Data that is meant to be immutable should be \code{const}
    (\href{https://isocpp.github.io/CppCoreGuidelines/CppCoreGuidelines#con4-use-const-to-define-objects-with-values-that-do-not-change-after-construction}{Con.4})
  \end{itemize}

  \begin{codeblock}<2->{
int \alert{const} x\{1\textquotesingle000\textquotesingle000\textquotesingle000\}; // or const int
\uncover<3->{std::cout << x + 32;    // ok, read-only}
\uncover<4->{x += 32;                // error, trying to modify}
\uncover<5->{int const y;            // error, not initialized and not modifiable later}
}\end{codeblock}

\begin{codeblock}<6->{
std::string \alert{const} message\{"Hello"\};
\uncover<7->{std::cout << message + " Francesco";      // ok, read-only}
\uncover<8->{message += " Francesco";                  // error, trying to modify}
\uncover<9->{std::string const empty_message;          // ok! empty string}}\end{codeblock}

  \begin{itemize}[<10->]
  \item Primitive types (e.g. \code{int}) and user-defined types such as
    \code{std::string}) behave differently with respect to \textit{default
      initialization}, i.e. without an explicit initial value
    \begin{itemize}
    \item User-defined types can define what default initialization means
    \item For \code{std::string} it means ``empty string''
    \item More later
    \end{itemize}
  \end{itemize}

\end{frame}
